\chapter{EEG信号采集模块的设计与实现}
\echapter{Design and Implementation fo the EEG Acqusition Module}

    \section{引言}
    \esection{Introduction}
        EEG信号具有信号幅度小、检测难度大的特性。这些特性对放大器、电源质量、
        信号滤波均提出了很大的要求。如果从基础的放大器开始搭建会较为困难,而且难以
        在满足便携式的要求下进行多个导联的同时采集。为了克服从放大器搭建的这些缺点
        我们的设计主要是基于模拟前端集成芯片来实现,这样便于兼顾便携性、信号质量
        以及导联数量的要求。我们在整个脑电采集的过程中运用了模块化设计的思想:因此
        ,第一步的设计主要在搭载模拟前端芯片的EEG信号采集模块的PCB板,后续调试
        通过后再以此为基础设计信号处理、传输控制部分等硬件电路。\par
        \vspace{1 ex}
        本章将主要描述作者对EEG信号采集模块的实现的过程,将会主要从以下几方面
        探讨:1)模块总体的结构和设计思路;2)模块各个子系统的设计细节;3)模块设计
        中遇到的问题以及解决。\par

        

        

    \section{EEG信号采集模块整体结构}
    \esection{Overall Struceure of the EEG Acqusition Module}

    \section{EEG采集模块设计细节与原理}
    \esection{Design Details and Principles of the EEG Acqusition Module}

        \subsection{EEG信号采集探头}
        \esection{EEG Siginal Acqusition Probe}

        \subsection{EEG信号采集、放大核心芯片}
        \esection{EEG signal acqusition and amplify IC}

        \subsection{EEG采集模块电源系统设计}
        \esection{Power SUpply Design of the EEG Acqusition Module}

        \subsection{EEG采集模块隔离与保护系统设计}
        \esection{Isolation and Protection Design of the EEG Acqusition Module}

    \section{本章小结}
    \esection{Brief Summary}